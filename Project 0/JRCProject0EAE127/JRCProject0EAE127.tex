\documentclass[11pt]{article}

    \usepackage[breakable]{tcolorbox}
    \usepackage{parskip} % Stop auto-indenting (to mimic markdown behaviour)
    
    \usepackage{iftex}
    \ifPDFTeX
    	\usepackage[T1]{fontenc}
    	\usepackage{mathpazo}
    \else
    	\usepackage{fontspec}
    \fi

    % Basic figure setup, for now with no caption control since it's done
    % automatically by Pandoc (which extracts ![](path) syntax from Markdown).
    \usepackage{graphicx}
    % Maintain compatibility with old templates. Remove in nbconvert 6.0
    \let\Oldincludegraphics\includegraphics
    % Ensure that by default, figures have no caption (until we provide a
    % proper Figure object with a Caption API and a way to capture that
    % in the conversion process - todo).
    \usepackage{caption}
    \DeclareCaptionFormat{nocaption}{}
    \captionsetup{format=nocaption,aboveskip=0pt,belowskip=0pt}

    \usepackage{float}
    \floatplacement{figure}{H} % forces figures to be placed at the correct location
    \usepackage{xcolor} % Allow colors to be defined
    \usepackage{enumerate} % Needed for markdown enumerations to work
    \usepackage{geometry} % Used to adjust the document margins
    \usepackage{amsmath} % Equations
    \usepackage{amssymb} % Equations
    \usepackage{textcomp} % defines textquotesingle
    % Hack from http://tex.stackexchange.com/a/47451/13684:
    \AtBeginDocument{%
        \def\PYZsq{\textquotesingle}% Upright quotes in Pygmentized code
    }
    \usepackage{upquote} % Upright quotes for verbatim code
    \usepackage{eurosym} % defines \euro
    \usepackage[mathletters]{ucs} % Extended unicode (utf-8) support
    \usepackage{fancyvrb} % verbatim replacement that allows latex
    \usepackage{grffile} % extends the file name processing of package graphics 
                         % to support a larger range
    \makeatletter % fix for old versions of grffile with XeLaTeX
    \@ifpackagelater{grffile}{2019/11/01}
    {
      % Do nothing on new versions
    }
    {
      \def\Gread@@xetex#1{%
        \IfFileExists{"\Gin@base".bb}%
        {\Gread@eps{\Gin@base.bb}}%
        {\Gread@@xetex@aux#1}%
      }
    }
    \makeatother
    \usepackage[Export]{adjustbox} % Used to constrain images to a maximum size
    \adjustboxset{max size={0.9\linewidth}{0.9\paperheight}}

    % The hyperref package gives us a pdf with properly built
    % internal navigation ('pdf bookmarks' for the table of contents,
    % internal cross-reference links, web links for URLs, etc.)
    \usepackage{hyperref}
    % The default LaTeX title has an obnoxious amount of whitespace. By default,
    % titling removes some of it. It also provides customization options.
    \usepackage{titling}
    \usepackage{longtable} % longtable support required by pandoc >1.10
    \usepackage{booktabs}  % table support for pandoc > 1.12.2
    \usepackage[inline]{enumitem} % IRkernel/repr support (it uses the enumerate* environment)
    \usepackage[normalem]{ulem} % ulem is needed to support strikethroughs (\sout)
                                % normalem makes italics be italics, not underlines
    \usepackage{mathrsfs}
    

    
    % Colors for the hyperref package
    \definecolor{urlcolor}{rgb}{0,.145,.698}
    \definecolor{linkcolor}{rgb}{.71,0.21,0.01}
    \definecolor{citecolor}{rgb}{.12,.54,.11}

    % ANSI colors
    \definecolor{ansi-black}{HTML}{3E424D}
    \definecolor{ansi-black-intense}{HTML}{282C36}
    \definecolor{ansi-red}{HTML}{E75C58}
    \definecolor{ansi-red-intense}{HTML}{B22B31}
    \definecolor{ansi-green}{HTML}{00A250}
    \definecolor{ansi-green-intense}{HTML}{007427}
    \definecolor{ansi-yellow}{HTML}{DDB62B}
    \definecolor{ansi-yellow-intense}{HTML}{B27D12}
    \definecolor{ansi-blue}{HTML}{208FFB}
    \definecolor{ansi-blue-intense}{HTML}{0065CA}
    \definecolor{ansi-magenta}{HTML}{D160C4}
    \definecolor{ansi-magenta-intense}{HTML}{A03196}
    \definecolor{ansi-cyan}{HTML}{60C6C8}
    \definecolor{ansi-cyan-intense}{HTML}{258F8F}
    \definecolor{ansi-white}{HTML}{C5C1B4}
    \definecolor{ansi-white-intense}{HTML}{A1A6B2}
    \definecolor{ansi-default-inverse-fg}{HTML}{FFFFFF}
    \definecolor{ansi-default-inverse-bg}{HTML}{000000}

    % common color for the border for error outputs.
    \definecolor{outerrorbackground}{HTML}{FFDFDF}

    % commands and environments needed by pandoc snippets
    % extracted from the output of `pandoc -s`
    \providecommand{\tightlist}{%
      \setlength{\itemsep}{0pt}\setlength{\parskip}{0pt}}
    \DefineVerbatimEnvironment{Highlighting}{Verbatim}{commandchars=\\\{\}}
    % Add ',fontsize=\small' for more characters per line
    \newenvironment{Shaded}{}{}
    \newcommand{\KeywordTok}[1]{\textcolor[rgb]{0.00,0.44,0.13}{\textbf{{#1}}}}
    \newcommand{\DataTypeTok}[1]{\textcolor[rgb]{0.56,0.13,0.00}{{#1}}}
    \newcommand{\DecValTok}[1]{\textcolor[rgb]{0.25,0.63,0.44}{{#1}}}
    \newcommand{\BaseNTok}[1]{\textcolor[rgb]{0.25,0.63,0.44}{{#1}}}
    \newcommand{\FloatTok}[1]{\textcolor[rgb]{0.25,0.63,0.44}{{#1}}}
    \newcommand{\CharTok}[1]{\textcolor[rgb]{0.25,0.44,0.63}{{#1}}}
    \newcommand{\StringTok}[1]{\textcolor[rgb]{0.25,0.44,0.63}{{#1}}}
    \newcommand{\CommentTok}[1]{\textcolor[rgb]{0.38,0.63,0.69}{\textit{{#1}}}}
    \newcommand{\OtherTok}[1]{\textcolor[rgb]{0.00,0.44,0.13}{{#1}}}
    \newcommand{\AlertTok}[1]{\textcolor[rgb]{1.00,0.00,0.00}{\textbf{{#1}}}}
    \newcommand{\FunctionTok}[1]{\textcolor[rgb]{0.02,0.16,0.49}{{#1}}}
    \newcommand{\RegionMarkerTok}[1]{{#1}}
    \newcommand{\ErrorTok}[1]{\textcolor[rgb]{1.00,0.00,0.00}{\textbf{{#1}}}}
    \newcommand{\NormalTok}[1]{{#1}}
    
    % Additional commands for more recent versions of Pandoc
    \newcommand{\ConstantTok}[1]{\textcolor[rgb]{0.53,0.00,0.00}{{#1}}}
    \newcommand{\SpecialCharTok}[1]{\textcolor[rgb]{0.25,0.44,0.63}{{#1}}}
    \newcommand{\VerbatimStringTok}[1]{\textcolor[rgb]{0.25,0.44,0.63}{{#1}}}
    \newcommand{\SpecialStringTok}[1]{\textcolor[rgb]{0.73,0.40,0.53}{{#1}}}
    \newcommand{\ImportTok}[1]{{#1}}
    \newcommand{\DocumentationTok}[1]{\textcolor[rgb]{0.73,0.13,0.13}{\textit{{#1}}}}
    \newcommand{\AnnotationTok}[1]{\textcolor[rgb]{0.38,0.63,0.69}{\textbf{\textit{{#1}}}}}
    \newcommand{\CommentVarTok}[1]{\textcolor[rgb]{0.38,0.63,0.69}{\textbf{\textit{{#1}}}}}
    \newcommand{\VariableTok}[1]{\textcolor[rgb]{0.10,0.09,0.49}{{#1}}}
    \newcommand{\ControlFlowTok}[1]{\textcolor[rgb]{0.00,0.44,0.13}{\textbf{{#1}}}}
    \newcommand{\OperatorTok}[1]{\textcolor[rgb]{0.40,0.40,0.40}{{#1}}}
    \newcommand{\BuiltInTok}[1]{{#1}}
    \newcommand{\ExtensionTok}[1]{{#1}}
    \newcommand{\PreprocessorTok}[1]{\textcolor[rgb]{0.74,0.48,0.00}{{#1}}}
    \newcommand{\AttributeTok}[1]{\textcolor[rgb]{0.49,0.56,0.16}{{#1}}}
    \newcommand{\InformationTok}[1]{\textcolor[rgb]{0.38,0.63,0.69}{\textbf{\textit{{#1}}}}}
    \newcommand{\WarningTok}[1]{\textcolor[rgb]{0.38,0.63,0.69}{\textbf{\textit{{#1}}}}}
    
    
    % Define a nice break command that doesn't care if a line doesn't already
    % exist.
    \def\br{\hspace*{\fill} \\* }
    % Math Jax compatibility definitions
    \def\gt{>}
    \def\lt{<}
    \let\Oldtex\TeX
    \let\Oldlatex\LaTeX
    \renewcommand{\TeX}{\textrm{\Oldtex}}
    \renewcommand{\LaTeX}{\textrm{\Oldlatex}}
    % Document parameters
    % Document title
    \title{JRCProject0EAE127}
    
    
    
    
    
% Pygments definitions
\makeatletter
\def\PY@reset{\let\PY@it=\relax \let\PY@bf=\relax%
    \let\PY@ul=\relax \let\PY@tc=\relax%
    \let\PY@bc=\relax \let\PY@ff=\relax}
\def\PY@tok#1{\csname PY@tok@#1\endcsname}
\def\PY@toks#1+{\ifx\relax#1\empty\else%
    \PY@tok{#1}\expandafter\PY@toks\fi}
\def\PY@do#1{\PY@bc{\PY@tc{\PY@ul{%
    \PY@it{\PY@bf{\PY@ff{#1}}}}}}}
\def\PY#1#2{\PY@reset\PY@toks#1+\relax+\PY@do{#2}}

\expandafter\def\csname PY@tok@w\endcsname{\def\PY@tc##1{\textcolor[rgb]{0.73,0.73,0.73}{##1}}}
\expandafter\def\csname PY@tok@c\endcsname{\let\PY@it=\textit\def\PY@tc##1{\textcolor[rgb]{0.25,0.50,0.50}{##1}}}
\expandafter\def\csname PY@tok@cp\endcsname{\def\PY@tc##1{\textcolor[rgb]{0.74,0.48,0.00}{##1}}}
\expandafter\def\csname PY@tok@k\endcsname{\let\PY@bf=\textbf\def\PY@tc##1{\textcolor[rgb]{0.00,0.50,0.00}{##1}}}
\expandafter\def\csname PY@tok@kp\endcsname{\def\PY@tc##1{\textcolor[rgb]{0.00,0.50,0.00}{##1}}}
\expandafter\def\csname PY@tok@kt\endcsname{\def\PY@tc##1{\textcolor[rgb]{0.69,0.00,0.25}{##1}}}
\expandafter\def\csname PY@tok@o\endcsname{\def\PY@tc##1{\textcolor[rgb]{0.40,0.40,0.40}{##1}}}
\expandafter\def\csname PY@tok@ow\endcsname{\let\PY@bf=\textbf\def\PY@tc##1{\textcolor[rgb]{0.67,0.13,1.00}{##1}}}
\expandafter\def\csname PY@tok@nb\endcsname{\def\PY@tc##1{\textcolor[rgb]{0.00,0.50,0.00}{##1}}}
\expandafter\def\csname PY@tok@nf\endcsname{\def\PY@tc##1{\textcolor[rgb]{0.00,0.00,1.00}{##1}}}
\expandafter\def\csname PY@tok@nc\endcsname{\let\PY@bf=\textbf\def\PY@tc##1{\textcolor[rgb]{0.00,0.00,1.00}{##1}}}
\expandafter\def\csname PY@tok@nn\endcsname{\let\PY@bf=\textbf\def\PY@tc##1{\textcolor[rgb]{0.00,0.00,1.00}{##1}}}
\expandafter\def\csname PY@tok@ne\endcsname{\let\PY@bf=\textbf\def\PY@tc##1{\textcolor[rgb]{0.82,0.25,0.23}{##1}}}
\expandafter\def\csname PY@tok@nv\endcsname{\def\PY@tc##1{\textcolor[rgb]{0.10,0.09,0.49}{##1}}}
\expandafter\def\csname PY@tok@no\endcsname{\def\PY@tc##1{\textcolor[rgb]{0.53,0.00,0.00}{##1}}}
\expandafter\def\csname PY@tok@nl\endcsname{\def\PY@tc##1{\textcolor[rgb]{0.63,0.63,0.00}{##1}}}
\expandafter\def\csname PY@tok@ni\endcsname{\let\PY@bf=\textbf\def\PY@tc##1{\textcolor[rgb]{0.60,0.60,0.60}{##1}}}
\expandafter\def\csname PY@tok@na\endcsname{\def\PY@tc##1{\textcolor[rgb]{0.49,0.56,0.16}{##1}}}
\expandafter\def\csname PY@tok@nt\endcsname{\let\PY@bf=\textbf\def\PY@tc##1{\textcolor[rgb]{0.00,0.50,0.00}{##1}}}
\expandafter\def\csname PY@tok@nd\endcsname{\def\PY@tc##1{\textcolor[rgb]{0.67,0.13,1.00}{##1}}}
\expandafter\def\csname PY@tok@s\endcsname{\def\PY@tc##1{\textcolor[rgb]{0.73,0.13,0.13}{##1}}}
\expandafter\def\csname PY@tok@sd\endcsname{\let\PY@it=\textit\def\PY@tc##1{\textcolor[rgb]{0.73,0.13,0.13}{##1}}}
\expandafter\def\csname PY@tok@si\endcsname{\let\PY@bf=\textbf\def\PY@tc##1{\textcolor[rgb]{0.73,0.40,0.53}{##1}}}
\expandafter\def\csname PY@tok@se\endcsname{\let\PY@bf=\textbf\def\PY@tc##1{\textcolor[rgb]{0.73,0.40,0.13}{##1}}}
\expandafter\def\csname PY@tok@sr\endcsname{\def\PY@tc##1{\textcolor[rgb]{0.73,0.40,0.53}{##1}}}
\expandafter\def\csname PY@tok@ss\endcsname{\def\PY@tc##1{\textcolor[rgb]{0.10,0.09,0.49}{##1}}}
\expandafter\def\csname PY@tok@sx\endcsname{\def\PY@tc##1{\textcolor[rgb]{0.00,0.50,0.00}{##1}}}
\expandafter\def\csname PY@tok@m\endcsname{\def\PY@tc##1{\textcolor[rgb]{0.40,0.40,0.40}{##1}}}
\expandafter\def\csname PY@tok@gh\endcsname{\let\PY@bf=\textbf\def\PY@tc##1{\textcolor[rgb]{0.00,0.00,0.50}{##1}}}
\expandafter\def\csname PY@tok@gu\endcsname{\let\PY@bf=\textbf\def\PY@tc##1{\textcolor[rgb]{0.50,0.00,0.50}{##1}}}
\expandafter\def\csname PY@tok@gd\endcsname{\def\PY@tc##1{\textcolor[rgb]{0.63,0.00,0.00}{##1}}}
\expandafter\def\csname PY@tok@gi\endcsname{\def\PY@tc##1{\textcolor[rgb]{0.00,0.63,0.00}{##1}}}
\expandafter\def\csname PY@tok@gr\endcsname{\def\PY@tc##1{\textcolor[rgb]{1.00,0.00,0.00}{##1}}}
\expandafter\def\csname PY@tok@ge\endcsname{\let\PY@it=\textit}
\expandafter\def\csname PY@tok@gs\endcsname{\let\PY@bf=\textbf}
\expandafter\def\csname PY@tok@gp\endcsname{\let\PY@bf=\textbf\def\PY@tc##1{\textcolor[rgb]{0.00,0.00,0.50}{##1}}}
\expandafter\def\csname PY@tok@go\endcsname{\def\PY@tc##1{\textcolor[rgb]{0.53,0.53,0.53}{##1}}}
\expandafter\def\csname PY@tok@gt\endcsname{\def\PY@tc##1{\textcolor[rgb]{0.00,0.27,0.87}{##1}}}
\expandafter\def\csname PY@tok@err\endcsname{\def\PY@bc##1{\setlength{\fboxsep}{0pt}\fcolorbox[rgb]{1.00,0.00,0.00}{1,1,1}{\strut ##1}}}
\expandafter\def\csname PY@tok@kc\endcsname{\let\PY@bf=\textbf\def\PY@tc##1{\textcolor[rgb]{0.00,0.50,0.00}{##1}}}
\expandafter\def\csname PY@tok@kd\endcsname{\let\PY@bf=\textbf\def\PY@tc##1{\textcolor[rgb]{0.00,0.50,0.00}{##1}}}
\expandafter\def\csname PY@tok@kn\endcsname{\let\PY@bf=\textbf\def\PY@tc##1{\textcolor[rgb]{0.00,0.50,0.00}{##1}}}
\expandafter\def\csname PY@tok@kr\endcsname{\let\PY@bf=\textbf\def\PY@tc##1{\textcolor[rgb]{0.00,0.50,0.00}{##1}}}
\expandafter\def\csname PY@tok@bp\endcsname{\def\PY@tc##1{\textcolor[rgb]{0.00,0.50,0.00}{##1}}}
\expandafter\def\csname PY@tok@fm\endcsname{\def\PY@tc##1{\textcolor[rgb]{0.00,0.00,1.00}{##1}}}
\expandafter\def\csname PY@tok@vc\endcsname{\def\PY@tc##1{\textcolor[rgb]{0.10,0.09,0.49}{##1}}}
\expandafter\def\csname PY@tok@vg\endcsname{\def\PY@tc##1{\textcolor[rgb]{0.10,0.09,0.49}{##1}}}
\expandafter\def\csname PY@tok@vi\endcsname{\def\PY@tc##1{\textcolor[rgb]{0.10,0.09,0.49}{##1}}}
\expandafter\def\csname PY@tok@vm\endcsname{\def\PY@tc##1{\textcolor[rgb]{0.10,0.09,0.49}{##1}}}
\expandafter\def\csname PY@tok@sa\endcsname{\def\PY@tc##1{\textcolor[rgb]{0.73,0.13,0.13}{##1}}}
\expandafter\def\csname PY@tok@sb\endcsname{\def\PY@tc##1{\textcolor[rgb]{0.73,0.13,0.13}{##1}}}
\expandafter\def\csname PY@tok@sc\endcsname{\def\PY@tc##1{\textcolor[rgb]{0.73,0.13,0.13}{##1}}}
\expandafter\def\csname PY@tok@dl\endcsname{\def\PY@tc##1{\textcolor[rgb]{0.73,0.13,0.13}{##1}}}
\expandafter\def\csname PY@tok@s2\endcsname{\def\PY@tc##1{\textcolor[rgb]{0.73,0.13,0.13}{##1}}}
\expandafter\def\csname PY@tok@sh\endcsname{\def\PY@tc##1{\textcolor[rgb]{0.73,0.13,0.13}{##1}}}
\expandafter\def\csname PY@tok@s1\endcsname{\def\PY@tc##1{\textcolor[rgb]{0.73,0.13,0.13}{##1}}}
\expandafter\def\csname PY@tok@mb\endcsname{\def\PY@tc##1{\textcolor[rgb]{0.40,0.40,0.40}{##1}}}
\expandafter\def\csname PY@tok@mf\endcsname{\def\PY@tc##1{\textcolor[rgb]{0.40,0.40,0.40}{##1}}}
\expandafter\def\csname PY@tok@mh\endcsname{\def\PY@tc##1{\textcolor[rgb]{0.40,0.40,0.40}{##1}}}
\expandafter\def\csname PY@tok@mi\endcsname{\def\PY@tc##1{\textcolor[rgb]{0.40,0.40,0.40}{##1}}}
\expandafter\def\csname PY@tok@il\endcsname{\def\PY@tc##1{\textcolor[rgb]{0.40,0.40,0.40}{##1}}}
\expandafter\def\csname PY@tok@mo\endcsname{\def\PY@tc##1{\textcolor[rgb]{0.40,0.40,0.40}{##1}}}
\expandafter\def\csname PY@tok@ch\endcsname{\let\PY@it=\textit\def\PY@tc##1{\textcolor[rgb]{0.25,0.50,0.50}{##1}}}
\expandafter\def\csname PY@tok@cm\endcsname{\let\PY@it=\textit\def\PY@tc##1{\textcolor[rgb]{0.25,0.50,0.50}{##1}}}
\expandafter\def\csname PY@tok@cpf\endcsname{\let\PY@it=\textit\def\PY@tc##1{\textcolor[rgb]{0.25,0.50,0.50}{##1}}}
\expandafter\def\csname PY@tok@c1\endcsname{\let\PY@it=\textit\def\PY@tc##1{\textcolor[rgb]{0.25,0.50,0.50}{##1}}}
\expandafter\def\csname PY@tok@cs\endcsname{\let\PY@it=\textit\def\PY@tc##1{\textcolor[rgb]{0.25,0.50,0.50}{##1}}}

\def\PYZbs{\char`\\}
\def\PYZus{\char`\_}
\def\PYZob{\char`\{}
\def\PYZcb{\char`\}}
\def\PYZca{\char`\^}
\def\PYZam{\char`\&}
\def\PYZlt{\char`\<}
\def\PYZgt{\char`\>}
\def\PYZsh{\char`\#}
\def\PYZpc{\char`\%}
\def\PYZdl{\char`\$}
\def\PYZhy{\char`\-}
\def\PYZsq{\char`\'}
\def\PYZdq{\char`\"}
\def\PYZti{\char`\~}
% for compatibility with earlier versions
\def\PYZat{@}
\def\PYZlb{[}
\def\PYZrb{]}
\makeatother


    % For linebreaks inside Verbatim environment from package fancyvrb. 
    \makeatletter
        \newbox\Wrappedcontinuationbox 
        \newbox\Wrappedvisiblespacebox 
        \newcommand*\Wrappedvisiblespace {\textcolor{red}{\textvisiblespace}} 
        \newcommand*\Wrappedcontinuationsymbol {\textcolor{red}{\llap{\tiny$\m@th\hookrightarrow$}}} 
        \newcommand*\Wrappedcontinuationindent {3ex } 
        \newcommand*\Wrappedafterbreak {\kern\Wrappedcontinuationindent\copy\Wrappedcontinuationbox} 
        % Take advantage of the already applied Pygments mark-up to insert 
        % potential linebreaks for TeX processing. 
        %        {, <, #, %, $, ' and ": go to next line. 
        %        _, }, ^, &, >, - and ~: stay at end of broken line. 
        % Use of \textquotesingle for straight quote. 
        \newcommand*\Wrappedbreaksatspecials {% 
            \def\PYGZus{\discretionary{\char`\_}{\Wrappedafterbreak}{\char`\_}}% 
            \def\PYGZob{\discretionary{}{\Wrappedafterbreak\char`\{}{\char`\{}}% 
            \def\PYGZcb{\discretionary{\char`\}}{\Wrappedafterbreak}{\char`\}}}% 
            \def\PYGZca{\discretionary{\char`\^}{\Wrappedafterbreak}{\char`\^}}% 
            \def\PYGZam{\discretionary{\char`\&}{\Wrappedafterbreak}{\char`\&}}% 
            \def\PYGZlt{\discretionary{}{\Wrappedafterbreak\char`\<}{\char`\<}}% 
            \def\PYGZgt{\discretionary{\char`\>}{\Wrappedafterbreak}{\char`\>}}% 
            \def\PYGZsh{\discretionary{}{\Wrappedafterbreak\char`\#}{\char`\#}}% 
            \def\PYGZpc{\discretionary{}{\Wrappedafterbreak\char`\%}{\char`\%}}% 
            \def\PYGZdl{\discretionary{}{\Wrappedafterbreak\char`\$}{\char`\$}}% 
            \def\PYGZhy{\discretionary{\char`\-}{\Wrappedafterbreak}{\char`\-}}% 
            \def\PYGZsq{\discretionary{}{\Wrappedafterbreak\textquotesingle}{\textquotesingle}}% 
            \def\PYGZdq{\discretionary{}{\Wrappedafterbreak\char`\"}{\char`\"}}% 
            \def\PYGZti{\discretionary{\char`\~}{\Wrappedafterbreak}{\char`\~}}% 
        } 
        % Some characters . , ; ? ! / are not pygmentized. 
        % This macro makes them "active" and they will insert potential linebreaks 
        \newcommand*\Wrappedbreaksatpunct {% 
            \lccode`\~`\.\lowercase{\def~}{\discretionary{\hbox{\char`\.}}{\Wrappedafterbreak}{\hbox{\char`\.}}}% 
            \lccode`\~`\,\lowercase{\def~}{\discretionary{\hbox{\char`\,}}{\Wrappedafterbreak}{\hbox{\char`\,}}}% 
            \lccode`\~`\;\lowercase{\def~}{\discretionary{\hbox{\char`\;}}{\Wrappedafterbreak}{\hbox{\char`\;}}}% 
            \lccode`\~`\:\lowercase{\def~}{\discretionary{\hbox{\char`\:}}{\Wrappedafterbreak}{\hbox{\char`\:}}}% 
            \lccode`\~`\?\lowercase{\def~}{\discretionary{\hbox{\char`\?}}{\Wrappedafterbreak}{\hbox{\char`\?}}}% 
            \lccode`\~`\!\lowercase{\def~}{\discretionary{\hbox{\char`\!}}{\Wrappedafterbreak}{\hbox{\char`\!}}}% 
            \lccode`\~`\/\lowercase{\def~}{\discretionary{\hbox{\char`\/}}{\Wrappedafterbreak}{\hbox{\char`\/}}}% 
            \catcode`\.\active
            \catcode`\,\active 
            \catcode`\;\active
            \catcode`\:\active
            \catcode`\?\active
            \catcode`\!\active
            \catcode`\/\active 
            \lccode`\~`\~ 	
        }
    \makeatother

    \let\OriginalVerbatim=\Verbatim
    \makeatletter
    \renewcommand{\Verbatim}[1][1]{%
        %\parskip\z@skip
        \sbox\Wrappedcontinuationbox {\Wrappedcontinuationsymbol}%
        \sbox\Wrappedvisiblespacebox {\FV@SetupFont\Wrappedvisiblespace}%
        \def\FancyVerbFormatLine ##1{\hsize\linewidth
            \vtop{\raggedright\hyphenpenalty\z@\exhyphenpenalty\z@
                \doublehyphendemerits\z@\finalhyphendemerits\z@
                \strut ##1\strut}%
        }%
        % If the linebreak is at a space, the latter will be displayed as visible
        % space at end of first line, and a continuation symbol starts next line.
        % Stretch/shrink are however usually zero for typewriter font.
        \def\FV@Space {%
            \nobreak\hskip\z@ plus\fontdimen3\font minus\fontdimen4\font
            \discretionary{\copy\Wrappedvisiblespacebox}{\Wrappedafterbreak}
            {\kern\fontdimen2\font}%
        }%
        
        % Allow breaks at special characters using \PYG... macros.
        \Wrappedbreaksatspecials
        % Breaks at punctuation characters . , ; ? ! and / need catcode=\active 	
        \OriginalVerbatim[#1,codes*=\Wrappedbreaksatpunct]%
    }
    \makeatother

    % Exact colors from NB
    \definecolor{incolor}{HTML}{303F9F}
    \definecolor{outcolor}{HTML}{D84315}
    \definecolor{cellborder}{HTML}{CFCFCF}
    \definecolor{cellbackground}{HTML}{F7F7F7}
    
    % prompt
    \makeatletter
    \newcommand{\boxspacing}{\kern\kvtcb@left@rule\kern\kvtcb@boxsep}
    \makeatother
    \newcommand{\prompt}[4]{
        {\ttfamily\llap{{\color{#2}[#3]:\hspace{3pt}#4}}\vspace{-\baselineskip}}
    }
    

    
    % Prevent overflowing lines due to hard-to-break entities
    \sloppy 
    % Setup hyperref package
    \hypersetup{
      breaklinks=true,  % so long urls are correctly broken across lines
      colorlinks=true,
      urlcolor=urlcolor,
      linkcolor=linkcolor,
      citecolor=citecolor,
      }
    % Slightly bigger margins than the latex defaults
    
    \geometry{verbose,tmargin=1in,bmargin=1in,lmargin=1in,rmargin=1in}
    
    

\begin{document}
    
    \maketitle
    
    

    
    \hypertarget{project-0-introduction-to-aerodynamics-and-python}{%
\section{Project 0: Introduction to Aerodynamics and
Python}\label{project-0-introduction-to-aerodynamics-and-python}}

\hypertarget{author-jack-comey}{%
\paragraph{Author: Jack Comey}\label{author-jack-comey}}

\hypertarget{student-id-915323775}{%
\subparagraph{Student ID: 915323775}\label{student-id-915323775}}

\hypertarget{due-date-200-pm-monday-19-oct-2020}{%
\subparagraph{Due Date: 2:00 PM Monday 19 OCT
2020}\label{due-date-200-pm-monday-19-oct-2020}}

Code Imports and Basic Function Definition:

    \begin{tcolorbox}[breakable, size=fbox, boxrule=1pt, pad at break*=1mm,colback=cellbackground, colframe=cellborder]
\prompt{In}{incolor}{1}{\boxspacing}
\begin{Verbatim}[commandchars=\\\{\}]
\PY{k+kn}{import} \PY{n+nn}{numpy} \PY{k}{as} \PY{n+nn}{np}
\PY{k+kn}{import} \PY{n+nn}{matplotlib} \PY{k}{as} \PY{n+nn}{mpl}
\PY{k+kn}{import} \PY{n+nn}{matplotlib}\PY{n+nn}{.}\PY{n+nn}{pyplot} \PY{k}{as} \PY{n+nn}{plt}
\PY{k+kn}{import} \PY{n+nn}{pandas} \PY{k}{as} \PY{n+nn}{pd}
\PY{k+kn}{import} \PY{n+nn}{handcalcs}\PY{n+nn}{.}\PY{n+nn}{render}
\PY{n}{plt}\PY{o}{.}\PY{n}{style}\PY{o}{.}\PY{n}{use}\PY{p}{(}\PY{l+s+s2}{\PYZdq{}}\PY{l+s+s2}{classic}\PY{l+s+s2}{\PYZdq{}}\PY{p}{)}

\PY{o}{\PYZpc{}}\PY{k}{matplotlib} inline


\PY{c+c1}{\PYZsh{} From solution document, modified slightly.}

\PY{n}{params}\PY{o}{=}\PY{p}{\PYZob{}}\PY{c+c1}{\PYZsh{}FONT SIZES}
    \PY{l+s+s1}{\PYZsq{}}\PY{l+s+s1}{axes.labelsize}\PY{l+s+s1}{\PYZsq{}}\PY{p}{:}\PY{l+m+mi}{30}\PY{p}{,}\PY{c+c1}{\PYZsh{}Axis Labels}
    \PY{l+s+s1}{\PYZsq{}}\PY{l+s+s1}{axes.titlesize}\PY{l+s+s1}{\PYZsq{}}\PY{p}{:}\PY{l+m+mi}{30}\PY{p}{,}\PY{c+c1}{\PYZsh{}Title}
    \PY{l+s+s1}{\PYZsq{}}\PY{l+s+s1}{font.size}\PY{l+s+s1}{\PYZsq{}}\PY{p}{:}\PY{l+m+mi}{28}\PY{p}{,}\PY{c+c1}{\PYZsh{}Textbox}
    \PY{l+s+s1}{\PYZsq{}}\PY{l+s+s1}{xtick.labelsize}\PY{l+s+s1}{\PYZsq{}}\PY{p}{:}\PY{l+m+mi}{22}\PY{p}{,}\PY{c+c1}{\PYZsh{}Axis tick labels}
    \PY{l+s+s1}{\PYZsq{}}\PY{l+s+s1}{ytick.labelsize}\PY{l+s+s1}{\PYZsq{}}\PY{p}{:}\PY{l+m+mi}{22}\PY{p}{,}\PY{c+c1}{\PYZsh{}Axis tick labels}
    \PY{l+s+s1}{\PYZsq{}}\PY{l+s+s1}{legend.fontsize}\PY{l+s+s1}{\PYZsq{}}\PY{p}{:}\PY{l+m+mi}{24}\PY{p}{,}\PY{c+c1}{\PYZsh{}Legend font size}
    \PY{l+s+s1}{\PYZsq{}}\PY{l+s+s1}{font.family}\PY{l+s+s1}{\PYZsq{}}\PY{p}{:}\PY{l+s+s1}{\PYZsq{}}\PY{l+s+s1}{serif}\PY{l+s+s1}{\PYZsq{}}\PY{p}{,}
    \PY{l+s+s1}{\PYZsq{}}\PY{l+s+s1}{font.fantasy}\PY{l+s+s1}{\PYZsq{}}\PY{p}{:}\PY{l+s+s1}{\PYZsq{}}\PY{l+s+s1}{xkcd}\PY{l+s+s1}{\PYZsq{}}\PY{p}{,}
    \PY{l+s+s1}{\PYZsq{}}\PY{l+s+s1}{font.sans\PYZhy{}serif}\PY{l+s+s1}{\PYZsq{}}\PY{p}{:}\PY{l+s+s1}{\PYZsq{}}\PY{l+s+s1}{Helvetica}\PY{l+s+s1}{\PYZsq{}}\PY{p}{,}
    \PY{l+s+s1}{\PYZsq{}}\PY{l+s+s1}{font.monospace}\PY{l+s+s1}{\PYZsq{}}\PY{p}{:}\PY{l+s+s1}{\PYZsq{}}\PY{l+s+s1}{Courier}\PY{l+s+s1}{\PYZsq{}}\PY{p}{,}
    \PY{c+c1}{\PYZsh{}AXIS PROPERTIES}
    \PY{l+s+s1}{\PYZsq{}}\PY{l+s+s1}{axes.titlepad}\PY{l+s+s1}{\PYZsq{}}\PY{p}{:}\PY{l+m+mi}{2}\PY{o}{*}\PY{l+m+mf}{6.0}\PY{p}{,}\PY{c+c1}{\PYZsh{}title spacing from axis}
    \PY{l+s+s1}{\PYZsq{}}\PY{l+s+s1}{axes.grid}\PY{l+s+s1}{\PYZsq{}}\PY{p}{:}\PY{k+kc}{True}\PY{p}{,}\PY{c+c1}{\PYZsh{}grid on plot}
    \PY{l+s+s1}{\PYZsq{}}\PY{l+s+s1}{figure.figsize}\PY{l+s+s1}{\PYZsq{}}\PY{p}{:}\PY{p}{(}\PY{l+m+mi}{12}\PY{p}{,}\PY{l+m+mi}{12}\PY{p}{)}\PY{p}{,}\PY{c+c1}{\PYZsh{}square plots}
    \PY{l+s+s1}{\PYZsq{}}\PY{l+s+s1}{savefig.bbox}\PY{l+s+s1}{\PYZsq{}}\PY{p}{:}\PY{l+s+s1}{\PYZsq{}}\PY{l+s+s1}{tight}\PY{l+s+s1}{\PYZsq{}}\PY{p}{,}\PY{c+c1}{\PYZsh{}reduce whitespace in saved figures\PYZsh{}LEGEND PROPERTIES}
    \PY{l+s+s1}{\PYZsq{}}\PY{l+s+s1}{legend.framealpha}\PY{l+s+s1}{\PYZsq{}}\PY{p}{:}\PY{l+m+mf}{0.5}\PY{p}{,}
    \PY{l+s+s1}{\PYZsq{}}\PY{l+s+s1}{legend.fancybox}\PY{l+s+s1}{\PYZsq{}}\PY{p}{:}\PY{k+kc}{True}\PY{p}{,}
    \PY{l+s+s1}{\PYZsq{}}\PY{l+s+s1}{legend.frameon}\PY{l+s+s1}{\PYZsq{}}\PY{p}{:}\PY{k+kc}{True}\PY{p}{,}
    \PY{l+s+s1}{\PYZsq{}}\PY{l+s+s1}{legend.numpoints}\PY{l+s+s1}{\PYZsq{}}\PY{p}{:}\PY{l+m+mi}{1}\PY{p}{,}
    \PY{l+s+s1}{\PYZsq{}}\PY{l+s+s1}{legend.scatterpoints}\PY{l+s+s1}{\PYZsq{}}\PY{p}{:}\PY{l+m+mi}{1}\PY{p}{,}
    \PY{l+s+s1}{\PYZsq{}}\PY{l+s+s1}{legend.borderpad}\PY{l+s+s1}{\PYZsq{}}\PY{p}{:}\PY{l+m+mf}{0.1}\PY{p}{,}
    \PY{l+s+s1}{\PYZsq{}}\PY{l+s+s1}{legend.borderaxespad}\PY{l+s+s1}{\PYZsq{}}\PY{p}{:}\PY{l+m+mf}{0.1}\PY{p}{,}
    \PY{l+s+s1}{\PYZsq{}}\PY{l+s+s1}{legend.handletextpad}\PY{l+s+s1}{\PYZsq{}}\PY{p}{:}\PY{l+m+mf}{0.2}\PY{p}{,}
    \PY{l+s+s1}{\PYZsq{}}\PY{l+s+s1}{legend.handlelength}\PY{l+s+s1}{\PYZsq{}}\PY{p}{:}\PY{l+m+mf}{1.0}\PY{p}{,}
    \PY{l+s+s1}{\PYZsq{}}\PY{l+s+s1}{legend.labelspacing}\PY{l+s+s1}{\PYZsq{}}\PY{p}{:}\PY{l+m+mi}{0}\PY{p}{,}\PY{p}{\PYZcb{}}
\PY{n}{mpl}\PY{o}{.}\PY{n}{rcParams}\PY{o}{.}\PY{n}{update}\PY{p}{(}\PY{n}{params}\PY{p}{)}

\PY{c+c1}{\PYZsh{}\PYZpc{}\PYZpc{}\PYZsh{}\PYZsh{}\PYZsh{}\PYZsh{}\PYZsh{}\PYZsh{}\PYZsh{}\PYZsh{}\PYZsh{}\PYZsh{}\PYZsh{}\PYZsh{}\PYZsh{}\PYZsh{}\PYZsh{}\PYZsh{}\PYZsh{}\PYZsh{}\PYZsh{}\PYZsh{}\PYZsh{}\PYZsh{}\PYZsh{}\PYZsh{}\PYZsh{}\PYZsh{}\PYZsh{}}

\PY{c+c1}{\PYZsh{} Custom Functions}

\PY{l+s+sd}{\PYZdq{}\PYZdq{}\PYZdq{}}
\PY{l+s+sd}{I\PYZsq{}ve used python extensively since February, and have created}
\PY{l+s+sd}{several quality\PYZhy{}of\PYZhy{}life functions to simplify repeated blocks}
\PY{l+s+sd}{of code. These are two which simplify plotting using matplotlib\PYZsq{}s}
\PY{l+s+sd}{pyplot library.}
\PY{l+s+sd}{\PYZdq{}\PYZdq{}\PYZdq{}}

\PY{k}{def} \PY{n+nf}{plothusly}\PY{p}{(}\PY{n}{ax}\PY{p}{,} \PY{n}{x}\PY{p}{,} \PY{n}{y}\PY{p}{,} \PY{o}{*}\PY{p}{,} \PY{n}{xtitle}\PY{o}{=}\PY{l+s+s1}{\PYZsq{}}\PY{l+s+s1}{\PYZsq{}}\PY{p}{,} \PY{n}{ytitle}\PY{o}{=}\PY{l+s+s1}{\PYZsq{}}\PY{l+s+s1}{\PYZsq{}}\PY{p}{,}
              \PY{n}{datalabel}\PY{o}{=}\PY{l+s+s1}{\PYZsq{}}\PY{l+s+s1}{\PYZsq{}}\PY{p}{,} \PY{n}{title}\PY{o}{=}\PY{l+s+s1}{\PYZsq{}}\PY{l+s+s1}{\PYZsq{}}\PY{p}{,} \PY{n}{linestyle} \PY{o}{=} \PY{l+s+s1}{\PYZsq{}}\PY{l+s+s1}{\PYZhy{}}\PY{l+s+s1}{\PYZsq{}}\PY{p}{,}
              \PY{n}{marker} \PY{o}{=} \PY{l+s+s1}{\PYZsq{}}\PY{l+s+s1}{\PYZsq{}}\PY{p}{)}\PY{p}{:}
    \PY{l+s+sd}{\PYZdq{}\PYZdq{}\PYZdq{}}
\PY{l+s+sd}{    A little function to make graphing easier.}
\PY{l+s+sd}{    Creates a plot with titles and axis labels.}
\PY{l+s+sd}{    Adds a new line to a blank figure and labels it.}

\PY{l+s+sd}{    Parameters}
\PY{l+s+sd}{    \PYZhy{}\PYZhy{}\PYZhy{}\PYZhy{}\PYZhy{}\PYZhy{}\PYZhy{}\PYZhy{}\PYZhy{}\PYZhy{}}
\PY{l+s+sd}{    ax : The graph object}
\PY{l+s+sd}{    x : X axis data}
\PY{l+s+sd}{    y : Y axis data}
\PY{l+s+sd}{    xtitle : Optional x axis data title. The default is \PYZsq{}\PYZsq{}.}
\PY{l+s+sd}{    ytitle : Optional y axis data title. The default is \PYZsq{}\PYZsq{}.}
\PY{l+s+sd}{    datalabel : Optional label for data. The default is \PYZsq{}\PYZsq{}.}
\PY{l+s+sd}{    title : Graph Title. The default is \PYZsq{}\PYZsq{}.}

\PY{l+s+sd}{    Returns}
\PY{l+s+sd}{    \PYZhy{}\PYZhy{}\PYZhy{}\PYZhy{}\PYZhy{}\PYZhy{}\PYZhy{}}
\PY{l+s+sd}{    out : Resultant graph.}

\PY{l+s+sd}{    \PYZdq{}\PYZdq{}\PYZdq{}}

    \PY{n}{ax}\PY{o}{.}\PY{n}{set\PYZus{}xlabel}\PY{p}{(}\PY{n}{xtitle}\PY{p}{)}
    \PY{n}{ax}\PY{o}{.}\PY{n}{set\PYZus{}ylabel}\PY{p}{(}\PY{n}{ytitle}\PY{p}{)}
    \PY{n}{ax}\PY{o}{.}\PY{n}{set\PYZus{}title}\PY{p}{(}\PY{n}{title}\PY{p}{)}
    \PY{n}{out} \PY{o}{=} \PY{n}{ax}\PY{o}{.}\PY{n}{plot}\PY{p}{(}\PY{n}{x}\PY{p}{,} \PY{n}{y}\PY{p}{,} \PY{n}{zorder}\PY{o}{=}\PY{l+m+mi}{1}\PY{p}{,} \PY{n}{label}\PY{o}{=}\PY{n}{datalabel}\PY{p}{,} \PY{n}{linestyle} \PY{o}{=} \PY{n}{linestyle}\PY{p}{,}
                  \PY{n}{marker} \PY{o}{=} \PY{n}{marker}\PY{p}{)}
    \PY{k}{return} \PY{n}{out}


\PY{k}{def} \PY{n+nf}{plothus}\PY{p}{(}\PY{n}{ax}\PY{p}{,} \PY{n}{x}\PY{p}{,} \PY{n}{y}\PY{p}{,} \PY{o}{*}\PY{p}{,} \PY{n}{datalabel}\PY{o}{=}\PY{l+s+s1}{\PYZsq{}}\PY{l+s+s1}{\PYZsq{}}\PY{p}{,} \PY{n}{linestyle} \PY{o}{=} \PY{l+s+s1}{\PYZsq{}}\PY{l+s+s1}{\PYZhy{}}\PY{l+s+s1}{\PYZsq{}}\PY{p}{,}
              \PY{n}{marker} \PY{o}{=} \PY{l+s+s1}{\PYZsq{}}\PY{l+s+s1}{\PYZsq{}}\PY{p}{)}\PY{p}{:}
    \PY{l+s+sd}{\PYZdq{}\PYZdq{}\PYZdq{}}
\PY{l+s+sd}{    A little function to make graphing easier.}

\PY{l+s+sd}{    Adds a new line to a blank figure and labels it}
\PY{l+s+sd}{    \PYZdq{}\PYZdq{}\PYZdq{}}
    \PY{n}{out} \PY{o}{=} \PY{n}{ax}\PY{o}{.}\PY{n}{plot}\PY{p}{(}\PY{n}{x}\PY{p}{,} \PY{n}{y}\PY{p}{,} \PY{n}{zorder}\PY{o}{=}\PY{l+m+mi}{1}\PY{p}{,} \PY{n}{label}\PY{o}{=}\PY{n}{datalabel}\PY{p}{,} \PY{n}{linestyle} \PY{o}{=} \PY{n}{linestyle}\PY{p}{,}
                  \PY{n}{marker} \PY{o}{=} \PY{n}{marker}\PY{p}{)}
    \PY{k}{return} \PY{n}{out}
\end{Verbatim}
\end{tcolorbox}

    \hypertarget{boundary-layers-and-numeric-integration}{%
\subsection{1 \textbar{} Boundary Layers and Numeric
Integration}\label{boundary-layers-and-numeric-integration}}

\hypertarget{turbulent-boundary-layer-and-velocity-profile}{%
\subsubsection{1.1 \textbar{} Turbulent Boundary Layer and Velocity
Profile}\label{turbulent-boundary-layer-and-velocity-profile}}

\hypertarget{approach}{%
\paragraph{1.1.1 \textbar{} Approach}\label{approach}}

The primary task in \textbf{1.1} is to plot the non-dimensionalized form
of a turbulent boundary layer, using the equation provided in the
problem statement. Input \(\frac{y}{\delta}\) can be defined as a series
of points using numpy's linspace function. \(\frac{u}{u_e}\) can be
found through the equation listed in \textbf{1.1.2}. Using matplotlib's
pyplot library, and a custom function listed in the preamble, the data
can be plotted on an equally spaced axes, and the data axes labeled.

Once this has beem done, an unmarked vertical line is to be plotted
along the y-axis, and used to shade in the boundary layer. Arrows can
then be drawn, at equally spaced intervals along the y-axis.

\hypertarget{equations}{%
\paragraph{1.1.2 \textbar{} Equations}\label{equations}}

A non-dimensional fluid boundary layer is defined as the following:

\[\begin{equation}
    \frac{u}{u_e} \approx (\frac{y}{\delta})^{1/7}
\end{equation}\]

\hypertarget{code-and-results}{%
\paragraph{1.1.3 \textbar{} Code and Results}\label{code-and-results}}

    \begin{tcolorbox}[breakable, size=fbox, boxrule=1pt, pad at break*=1mm,colback=cellbackground, colframe=cellborder]
\prompt{In}{incolor}{2}{\boxspacing}
\begin{Verbatim}[commandchars=\\\{\}]
\PY{n}{ynon} \PY{o}{=} \PY{n}{np}\PY{o}{.}\PY{n}{linspace}\PY{p}{(}\PY{l+m+mi}{0}\PY{p}{,} \PY{l+m+mi}{1}\PY{p}{,} \PY{l+m+mi}{1000}\PY{p}{)}
\PY{n}{unon} \PY{o}{=} \PY{n}{ynon}\PY{o}{*}\PY{o}{*}\PY{p}{(}\PY{l+m+mi}{1}\PY{o}{/}\PY{l+m+mi}{7}\PY{p}{)}

\PY{n}{fig}\PY{p}{,} \PY{n}{prob1aplot} \PY{o}{=} \PY{n}{plt}\PY{o}{.}\PY{n}{subplots}\PY{p}{(}\PY{p}{)}
\PY{n}{plothusly}\PY{p}{(}\PY{n}{prob1aplot}\PY{p}{,}
          \PY{n}{unon}\PY{p}{,}
          \PY{n}{ynon}\PY{p}{,}
          \PY{n}{xtitle}\PY{o}{=}\PY{l+s+s1}{\PYZsq{}}\PY{l+s+s1}{\PYZdl{}u/u\PYZus{}e\PYZdl{}}\PY{l+s+s1}{\PYZsq{}}\PY{p}{,}
          \PY{n}{ytitle}\PY{o}{=}\PY{l+s+s1}{\PYZsq{}}\PY{l+s+s1}{y/\PYZdl{}}\PY{l+s+s1}{\PYZbs{}}\PY{l+s+s1}{delta\PYZdl{}}\PY{l+s+s1}{\PYZsq{}}\PY{p}{,}
          \PY{n}{datalabel}\PY{o}{=}\PY{l+s+s1}{\PYZsq{}}\PY{l+s+s1}{Boundary layer}\PY{l+s+s1}{\PYZsq{}}\PY{p}{,}
          \PY{n}{title}\PY{o}{=}\PY{l+s+s1}{\PYZsq{}}\PY{l+s+s1}{Boundary Layer Velocity Profile}\PY{l+s+s1}{\PYZsq{}}\PY{p}{,} 
          \PY{n}{linestyle}\PY{o}{=}\PY{l+s+s1}{\PYZsq{}}\PY{l+s+s1}{solid}\PY{l+s+s1}{\PYZsq{}}\PY{p}{)}

\PY{c+c1}{\PYZsh{} Shade in boundary layer}
\PY{n}{vline} \PY{o}{=} \PY{n}{ynon}\PY{o}{*}\PY{l+m+mi}{0}
\PY{n}{plothus}\PY{p}{(}\PY{n}{prob1aplot}\PY{p}{,} \PY{n}{unon}\PY{p}{,} \PY{n}{vline}\PY{p}{,} \PY{n}{linestyle}\PY{o}{=}\PY{l+s+s1}{\PYZsq{}}\PY{l+s+s1}{\PYZsq{}}\PY{p}{)}
\PY{n}{plt}\PY{o}{.}\PY{n}{fill\PYZus{}betweenx}\PY{p}{(}\PY{n}{ynon}\PY{p}{,} \PY{n}{vline}\PY{p}{,} \PY{n}{unon}\PY{p}{,} \PY{n}{facecolor}\PY{o}{=}\PY{l+s+s1}{\PYZsq{}}\PY{l+s+s1}{b}\PY{l+s+s1}{\PYZsq{}}\PY{p}{,} \PY{n}{alpha}\PY{o}{=}\PY{l+m+mf}{0.3}\PY{p}{)}
\PY{n}{plt}\PY{o}{.}\PY{n}{axis}\PY{p}{(}\PY{l+s+s1}{\PYZsq{}}\PY{l+s+s1}{equal}\PY{l+s+s1}{\PYZsq{}}\PY{p}{)}

\PY{c+c1}{\PYZsh{} Make Arrows}
\PY{n}{arrowwidth}\PY{p}{,} \PY{n}{arrowlength} \PY{o}{=} \PY{l+m+mf}{0.02}\PY{p}{,} \PY{l+m+mf}{0.02}

\PY{k}{for} \PY{n}{i} \PY{o+ow}{in} \PY{n+nb}{range}\PY{p}{(}\PY{l+m+mi}{0}\PY{p}{,} \PY{n+nb}{len}\PY{p}{(}\PY{n}{ynon}\PY{p}{)}\PY{p}{,} \PY{l+m+mi}{50}\PY{p}{)}\PY{p}{:}
    \PY{k}{if} \PY{n+nb}{abs}\PY{p}{(}\PY{n}{unon}\PY{p}{[}\PY{n}{i}\PY{p}{]}\PY{p}{)} \PY{o}{\PYZlt{}} \PY{n}{arrowlength}\PY{p}{:}
        \PY{n}{plt}\PY{o}{.}\PY{n}{plot}\PY{p}{(}\PY{p}{[}\PY{l+m+mi}{0}\PY{p}{,} \PY{n}{unon}\PY{p}{[}\PY{n}{i}\PY{p}{]}\PY{p}{]}\PY{p}{,} \PY{p}{[}\PY{n}{ynon}\PY{p}{[}\PY{n}{i}\PY{p}{]}\PY{p}{,} \PY{n}{ynon}\PY{p}{[}\PY{n}{i}\PY{p}{]}\PY{p}{]}\PY{p}{,} \PY{n}{color}\PY{o}{=}\PY{l+s+s1}{\PYZsq{}}\PY{l+s+s1}{b}\PY{l+s+s1}{\PYZsq{}}\PY{p}{)}
    \PY{k}{else}\PY{p}{:}
        \PY{n}{plt}\PY{o}{.}\PY{n}{arrow}\PY{p}{(}\PY{l+m+mi}{0}\PY{p}{,} \PY{n}{ynon}\PY{p}{[}\PY{n}{i}\PY{p}{]}\PY{p}{,} \PY{n}{unon}\PY{p}{[}\PY{n}{i}\PY{p}{]}\PY{o}{\PYZhy{}}\PY{n}{arrowlength}\PY{p}{,} \PY{l+m+mi}{0}\PY{p}{,} \PY{n}{head\PYZus{}width}\PY{o}{=}\PY{n}{arrowwidth}\PY{p}{,}
                  \PY{n}{head\PYZus{}length}\PY{o}{=}\PY{n}{arrowlength}\PY{p}{,} \PY{n}{color}\PY{o}{=}\PY{l+s+s1}{\PYZsq{}}\PY{l+s+s1}{b}\PY{l+s+s1}{\PYZsq{}}\PY{p}{,} \PY{n}{linewidth}\PY{o}{=}\PY{l+m+mi}{2}\PY{p}{,} \PY{n}{alpha}\PY{o}{=}\PY{l+m+mf}{0.4}\PY{p}{)}
\end{Verbatim}
\end{tcolorbox}

    \begin{center}
    \adjustimage{max size={0.9\linewidth}{0.9\paperheight}}{output_3_0.png}
    \end{center}
    { \hspace*{\fill} \\}
    
    \hypertarget{boundary-layer-thickness}{%
\subsubsection{1.2 \textbar{} Boundary Layer
Thickness}\label{boundary-layer-thickness}}

\hypertarget{approach}{%
\paragraph{1.2.1 \textbar{} Approach}\label{approach}}

The goal of problem 1.2 is to find the boundary layer thickness
\(\delta\) given specific properties and an equation listed in the
problem statement, and calculation of displacement thickness
\(\delta^*\) through numerical integration and a re-dimensionalization.

\(\delta(x)\) can be defined as an anonymous function, and then found
using \(x\) as an input. \(\frac{\delta^*}{\delta}\) can be found by
integrating \(1 - \frac{u}{u_e}\) with respect to \(\frac{y}{\delta}\).
The result can then be re-dimensionalized using the \(\delta(x)\).

The output data is then placed into an f-string, which allows for
variables to be placed into strings, and then printed to the console.

\hypertarget{equations}{%
\paragraph{1.2.2 \textbar{} Equations}\label{equations}}

Boundary layer thickness at \(x\) is given by the equation:

\[\begin{equation}
    \delta(x) = \frac{0.16x}{(Re_x)^{1/7}}
\end{equation}\]

Displacement thickness \(\delta^*\) can be found using:

\[\begin{equation}
    \frac{\delta^*}{\delta} = \int_0^1 (1-\frac{u}{u_e})\, d\frac{y}{\delta}
\end{equation}\] \#\#\#\# 1.2.3 \textbar{} Code and Results

    \begin{tcolorbox}[breakable, size=fbox, boxrule=1pt, pad at break*=1mm,colback=cellbackground, colframe=cellborder]
\prompt{In}{incolor}{3}{\boxspacing}
\begin{Verbatim}[commandchars=\\\{\}]
\PY{c+c1}{\PYZsh{} Problem 1b}

\PY{c+c1}{\PYZsh{} Input properties}
\PY{n}{Re} \PY{o}{=} \PY{l+m+mf}{1E8}  \PY{c+c1}{\PYZsh{} ndim}
\PY{n}{pos} \PY{o}{=} \PY{l+m+mi}{300}  \PY{c+c1}{\PYZsh{} ft}

\PY{c+c1}{\PYZsh{} Find non\PYZhy{}dim delta\PYZhy{}star}
\PY{n}{delta\PYZus{}star\PYZus{}non\PYZus{}dim} \PY{o}{=} \PY{n}{np}\PY{o}{.}\PY{n}{trapz}\PY{p}{(}\PY{l+m+mi}{1} \PY{o}{\PYZhy{}} \PY{n}{unon}\PY{p}{,} \PY{n}{ynon}\PY{p}{)}

\PY{c+c1}{\PYZsh{} Set delta(x) as an anonymous function, and solve for x}
\PY{n}{delta\PYZus{}formula} \PY{o}{=} \PY{k}{lambda} \PY{n}{x} \PY{p}{:} \PY{p}{(}\PY{l+m+mf}{0.16}\PY{o}{*}\PY{n}{x}\PY{p}{)} \PY{o}{/} \PY{p}{(}\PY{p}{(}\PY{n}{Re}\PY{o}{*}\PY{o}{*}\PY{p}{(}\PY{l+m+mi}{1}\PY{o}{/}\PY{l+m+mi}{7}\PY{p}{)}\PY{p}{)}\PY{p}{)}
\PY{n}{delta\PYZus{}point} \PY{o}{=} \PY{n}{delta\PYZus{}formula}\PY{p}{(}\PY{n}{pos}\PY{p}{)}

\PY{c+c1}{\PYZsh{} Dimensionalize delta\PYZhy{}star, and print}
\PY{n}{delta\PYZus{}star} \PY{o}{=} \PY{n}{delta\PYZus{}star\PYZus{}non\PYZus{}dim} \PY{o}{*} \PY{n}{delta\PYZus{}point}
\PY{n}{string1} \PY{o}{=} \PY{l+s+sa}{f}\PY{l+s+s1}{\PYZsq{}}\PY{l+s+s1}{Displacement thickness of }\PY{l+s+si}{\PYZob{}}\PY{n}{delta\PYZus{}star}\PY{o}{*}\PY{l+m+mi}{12}\PY{l+s+si}{\PYZcb{}}\PY{l+s+s1}{ in. for [Re=}\PY{l+s+si}{\PYZob{}}\PY{n}{Re}\PY{l+s+si}{:}\PY{l+s+s1}{1.1E}\PY{l+s+si}{\PYZcb{}}\PY{l+s+s1}{, L = }\PY{l+s+si}{\PYZob{}}\PY{n}{pos}\PY{l+s+si}{\PYZcb{}}\PY{l+s+s1}{ft]}\PY{l+s+s1}{\PYZsq{}}
\PY{n}{string2} \PY{o}{=} \PY{l+s+sa}{f}\PY{l+s+s1}{\PYZsq{}}\PY{l+s+s1}{Boundary layer thickness: }\PY{l+s+si}{\PYZob{}}\PY{n}{delta\PYZus{}point}\PY{o}{*}\PY{l+m+mi}{12}\PY{l+s+si}{\PYZcb{}}\PY{l+s+s1}{ in}\PY{l+s+s1}{\PYZsq{}}
\PY{n+nb}{print}\PY{p}{(}\PY{n}{string1}\PY{p}{)}
\PY{n+nb}{print}\PY{p}{(}\PY{n}{string2}\PY{p}{)}
\end{Verbatim}
\end{tcolorbox}

    \begin{Verbatim}[commandchars=\\\{\}]
Displacement thickness of 5.187717716704352 in. for [Re=1.0E+08, L = 300ft]
Boundary layer thickness: 41.45389476486636 in
    \end{Verbatim}

    \hypertarget{airfoil-plotting-and-line-integrals}{%
\subsection{2 \textbar{} Airfoil Plotting and Line
Integrals}\label{airfoil-plotting-and-line-integrals}}

\hypertarget{airfoil-plotting}{%
\subsubsection{2.1 \textbar{} Airfoil Plotting}\label{airfoil-plotting}}

\hypertarget{approach}{%
\paragraph{2.1.1 \textbar{} Approach}\label{approach}}

The goal of Problem 2.1 is to distinctly plot three seperate airfoils,
using files avaliable from a University of Illinois database. Each
airfoil file in the database is provided as a .dat file. While .csv
files are easily read by pandas, .dat files are more easily read by
numpy.

Each .dat file can be modified to remove the header from the file, and
then imported as an array. Each array is then placed into a pandas
dataframe in order to store the data with headers. While the last step
is not strictly necessary, it allows for easier access of data and
better code readability. The three chosen airfoils are the Selig 6063,
the Selig 8036, and the airfoil used on the Hawker Tempest at 37.5\%
semi-span. The three airfoils are then plotted using matplotlib's pyplot
library, and differentiated on the plot through differing colors and
linestyles. Each airfoil is also labeled on the plot itself.

\hypertarget{equations}{%
\paragraph{2.1.2 \textbar{} Equations}\label{equations}}

No equations were used in this section, as it is primarily a plotting
excersize.

\hypertarget{code-and-results}{%
\paragraph{2.1.3 \textbar{} Code and Results}\label{code-and-results}}

    \begin{tcolorbox}[breakable, size=fbox, boxrule=1pt, pad at break*=1mm,colback=cellbackground, colframe=cellborder]
\prompt{In}{incolor}{4}{\boxspacing}
\begin{Verbatim}[commandchars=\\\{\}]
\PY{c+c1}{\PYZsh{} Problem 2a}

\PY{n}{s6063} \PY{o}{=} \PY{n}{np}\PY{o}{.}\PY{n}{loadtxt}\PY{p}{(}\PY{l+s+s1}{\PYZsq{}}\PY{l+s+s1}{Data/s6063.dat}\PY{l+s+s1}{\PYZsq{}}\PY{p}{)}
\PY{n}{s8036} \PY{o}{=} \PY{n}{np}\PY{o}{.}\PY{n}{loadtxt}\PY{p}{(}\PY{l+s+s1}{\PYZsq{}}\PY{l+s+s1}{Data/s8036.dat}\PY{l+s+s1}{\PYZsq{}}\PY{p}{)}
\PY{n}{tempest} \PY{o}{=} \PY{n}{np}\PY{o}{.}\PY{n}{loadtxt}\PY{p}{(}\PY{l+s+s1}{\PYZsq{}}\PY{l+s+s1}{Data/tempest1.dat}\PY{l+s+s1}{\PYZsq{}}\PY{p}{)}
\PY{n}{s6063df} \PY{o}{=} \PY{n}{pd}\PY{o}{.}\PY{n}{DataFrame}\PY{p}{(}\PY{n}{s6063}\PY{p}{)}
\PY{n}{s8036df} \PY{o}{=} \PY{n}{pd}\PY{o}{.}\PY{n}{DataFrame}\PY{p}{(}\PY{n}{s8036}\PY{p}{)}
\PY{n}{tempestdf} \PY{o}{=} \PY{n}{pd}\PY{o}{.}\PY{n}{DataFrame}\PY{p}{(}\PY{n}{tempest}\PY{p}{)}


\PY{n}{fig}\PY{p}{,} \PY{n}{airfoilplot} \PY{o}{=} \PY{n}{plt}\PY{o}{.}\PY{n}{subplots}\PY{p}{(}\PY{p}{)}
\PY{n}{plothusly}\PY{p}{(}\PY{n}{airfoilplot}\PY{p}{,} \PY{n}{s6063df}\PY{p}{[}\PY{l+m+mi}{0}\PY{p}{]}\PY{p}{,} \PY{n}{s6063df}\PY{p}{[}\PY{l+m+mi}{1}\PY{p}{]}\PY{p}{,} \PY{n}{marker}\PY{o}{=}\PY{l+s+s1}{\PYZsq{}}\PY{l+s+s1}{\PYZsq{}}\PY{p}{,} \PY{n}{title} \PY{o}{=} \PY{l+s+s2}{\PYZdq{}}\PY{l+s+s2}{Airfoil Plotting Comparison}\PY{l+s+s2}{\PYZdq{}}\PY{p}{,} \PY{n}{linestyle}\PY{o}{=}\PY{l+s+s1}{\PYZsq{}}\PY{l+s+s1}{\PYZhy{}}\PY{l+s+s1}{\PYZsq{}}\PY{p}{,} \PY{n}{datalabel}\PY{o}{=}\PY{l+s+s1}{\PYZsq{}}\PY{l+s+s1}{Selig 6063}\PY{l+s+s1}{\PYZsq{}}\PY{p}{)}
\PY{n}{plothus}\PY{p}{(}\PY{n}{airfoilplot}\PY{p}{,} \PY{n}{s8036df}\PY{p}{[}\PY{l+m+mi}{0}\PY{p}{]}\PY{p}{,} \PY{n}{s8036df}\PY{p}{[}\PY{l+m+mi}{1}\PY{p}{]}\PY{p}{,} \PY{n}{marker}\PY{o}{=}\PY{l+s+s1}{\PYZsq{}}\PY{l+s+s1}{\PYZsq{}}\PY{p}{,} \PY{n}{linestyle}\PY{o}{=}\PY{l+s+s1}{\PYZsq{}}\PY{l+s+s1}{\PYZhy{}\PYZhy{}}\PY{l+s+s1}{\PYZsq{}}\PY{p}{,} \PY{n}{datalabel}\PY{o}{=}\PY{l+s+s1}{\PYZsq{}}\PY{l+s+s1}{Selig 8036}\PY{l+s+s1}{\PYZsq{}}\PY{p}{)}
\PY{n}{plothus}\PY{p}{(}\PY{n}{airfoilplot}\PY{p}{,} \PY{n}{tempestdf}\PY{p}{[}\PY{l+m+mi}{0}\PY{p}{]}\PY{p}{,} \PY{n}{tempestdf}\PY{p}{[}\PY{l+m+mi}{1}\PY{p}{]}\PY{p}{,} \PY{n}{marker}\PY{o}{=}\PY{l+s+s1}{\PYZsq{}}\PY{l+s+s1}{\PYZsq{}}\PY{p}{,} \PY{n}{linestyle}\PY{o}{=}\PY{l+s+s1}{\PYZsq{}}\PY{l+s+s1}{\PYZhy{}.}\PY{l+s+s1}{\PYZsq{}}\PY{p}{,} \PY{n}{datalabel}\PY{o}{=}\PY{l+s+s1}{\PYZsq{}}\PY{l+s+s1}{Hawker Tempest 37.5}\PY{l+s+s1}{\PYZpc{}}\PY{l+s+s1}{ Semi\PYZhy{}span}\PY{l+s+s1}{\PYZsq{}}\PY{p}{)}
\PY{n}{plt}\PY{o}{.}\PY{n}{legend}\PY{p}{(}\PY{n}{loc}\PY{o}{=}\PY{l+s+s1}{\PYZsq{}}\PY{l+s+s1}{best}\PY{l+s+s1}{\PYZsq{}}\PY{p}{)}
\PY{n}{plt}\PY{o}{.}\PY{n}{axis}\PY{p}{(}\PY{l+s+s1}{\PYZsq{}}\PY{l+s+s1}{equal}\PY{l+s+s1}{\PYZsq{}}\PY{p}{)}
\end{Verbatim}
\end{tcolorbox}

            \begin{tcolorbox}[breakable, size=fbox, boxrule=.5pt, pad at break*=1mm, opacityfill=0]
\prompt{Out}{outcolor}{4}{\boxspacing}
\begin{Verbatim}[commandchars=\\\{\}]
(0.0, 1.0, -0.08, 0.09999999999999999)
\end{Verbatim}
\end{tcolorbox}
        
    \begin{center}
    \adjustimage{max size={0.9\linewidth}{0.9\paperheight}}{output_7_1.png}
    \end{center}
    { \hspace*{\fill} \\}
    
    \hypertarget{cross-sectional-area-via-line-integration}{%
\subsubsection{2.2 \textbar{} Cross Sectional Area via Line
Integration}\label{cross-sectional-area-via-line-integration}}

\hypertarget{approach}{%
\paragraph{2.2.1 \textbar{} Approach}\label{approach}}

The primary goal in Problem 2.2 is to find the cross-sectional area of a
NACA 2412, with a chord length of 9.5 feet, through numeric integration.
The area of any closed polygon can be calculated using Green's theorem,
which defines the double integral as a line integral. As in the previous
problem, the data from the .dat file can be imported using numpy, after
removing the first line, and then transferred into a pandas Dataframe
for easier calculations. The airfoil can be dimensionalized by
multiplying each column by the chord length. The integral can then be
calculated using numpy's np.trapz function, which uses trapezoidal
integration.

\hypertarget{equations}{%
\paragraph{2.2.2 \textbar{} Equations}\label{equations}}

Green's theorem is defined as:

\[\begin{equation}
    \iint_D dA = \frac{1}{2} \onint(-ydx + xdy) = \oint_C x \, dy
\end{equation}\]

\hypertarget{code-and-results}{%
\paragraph{2.2.3 \textbar{} Code and Results}\label{code-and-results}}

    \begin{tcolorbox}[breakable, size=fbox, boxrule=1pt, pad at break*=1mm,colback=cellbackground, colframe=cellborder]
\prompt{In}{incolor}{5}{\boxspacing}
\begin{Verbatim}[commandchars=\\\{\}]
\PY{c+c1}{\PYZsh{} Problem 2b}


\PY{n}{c} \PY{o}{=} \PY{l+m+mf}{9.5}  \PY{c+c1}{\PYZsh{} ft, chord length}
\PY{n}{naca\PYZus{}2412\PYZus{}data} \PY{o}{=} \PY{n}{np}\PY{o}{.}\PY{n}{loadtxt}\PY{p}{(}\PY{l+s+s1}{\PYZsq{}}\PY{l+s+s1}{Data/naca2412\PYZus{}geom.dat}\PY{l+s+s1}{\PYZsq{}}\PY{p}{,} \PY{n}{unpack}\PY{o}{=}\PY{k+kc}{True}\PY{p}{,} \PY{n}{skiprows}\PY{o}{=}\PY{l+m+mi}{1}\PY{p}{)}
\PY{n}{naca\PYZus{}2412\PYZus{}data} \PY{o}{*}\PY{o}{=} \PY{n}{c}  \PY{c+c1}{\PYZsh{} Redimensionalize}
\PY{n}{naca\PYZus{}2412\PYZus{}df} \PY{o}{=} \PY{n}{pd}\PY{o}{.}\PY{n}{DataFrame}\PY{p}{(}\PY{n}{naca\PYZus{}2412\PYZus{}data}\PY{o}{.}\PY{n}{transpose}\PY{p}{(}\PY{p}{)}\PY{p}{)}
\PY{n}{area} \PY{o}{=} \PY{n}{np}\PY{o}{.}\PY{n}{trapz}\PY{p}{(}\PY{n}{naca\PYZus{}2412\PYZus{}df}\PY{p}{[}\PY{l+m+mi}{0}\PY{p}{]}\PY{p}{,} \PY{n}{naca\PYZus{}2412\PYZus{}df}\PY{p}{[}\PY{l+m+mi}{1}\PY{p}{]}\PY{p}{)}
\PY{n}{string} \PY{o}{=} \PY{l+s+sa}{f}\PY{l+s+s2}{\PYZdq{}\PYZdq{}\PYZdq{}}\PY{l+s+s2}{Cross\PYZhy{}sectional area of NACA 2412 airfoil}
\PY{l+s+s2}{with chord length }\PY{l+s+si}{\PYZob{}}\PY{n}{c}\PY{l+s+si}{\PYZcb{}}\PY{l+s+s2}{ is }\PY{l+s+si}{\PYZob{}}\PY{n}{area}\PY{l+s+si}{\PYZcb{}}\PY{l+s+s2}{ square feet}\PY{l+s+s2}{\PYZdq{}\PYZdq{}\PYZdq{}}
\PY{n+nb}{print}\PY{p}{(}\PY{n}{string}\PY{p}{)}
\end{Verbatim}
\end{tcolorbox}

    \begin{Verbatim}[commandchars=\\\{\}]
Cross-sectional area of NACA 2412 airfoil
with chord length 9.5 is 7.190500819217841 square feet
    \end{Verbatim}

    \hypertarget{airfoil-surface-pressure-and-numeric-differentiation}{%
\subsection{3 \textbar{} Airfoil Surface Pressure and Numeric
Differentiation}\label{airfoil-surface-pressure-and-numeric-differentiation}}

\hypertarget{airfoil-surface-pressure}{%
\subsubsection{3.1 \textbar{} Airfoil Surface
Pressure}\label{airfoil-surface-pressure}}

\hypertarget{approach}{%
\paragraph{3.1.1 \textbar{} Approach}\label{approach}}

The goal of Problem 3.1 is to plot the nondimensional surface pressure
coefficient of a NACA 2412 airfoil, given a .csv file containing the
data. Using pandas' read\_csv() function, surface pressure coefficients
and their corresponding positional data can be placed into a pandas
dataframe directly from the file. To better visualize the effect of the
pressure, the negative of the pressure constant C\(_P\) is plotted,
using matplotlib's pyplot library. Both the upper and lower curves are
to be plotted as seperate curves, and labeled accordingly.

\hypertarget{equations}{%
\paragraph{3.1.2 \textbar{} Equations}\label{equations}}

No equations were used in this problem, as it is primarily a plotting
excersize.

\hypertarget{code-and-results}{%
\paragraph{3.1.3 \textbar{} Code and Results}\label{code-and-results}}

    \begin{tcolorbox}[breakable, size=fbox, boxrule=1pt, pad at break*=1mm,colback=cellbackground, colframe=cellborder]
\prompt{In}{incolor}{6}{\boxspacing}
\begin{Verbatim}[commandchars=\\\{\}]
\PY{c+c1}{\PYZsh{} Problem 3a }

\PY{c+c1}{\PYZsh{} Read the data}
\PY{n}{pressure\PYZus{}distribution} \PY{o}{=} \PY{n}{pd}\PY{o}{.}\PY{n}{read\PYZus{}csv}\PY{p}{(}\PY{l+s+s1}{\PYZsq{}}\PY{l+s+s1}{Data/naca2412\PYZus{}SurfPress\PYZus{}a6.csv}\PY{l+s+s1}{\PYZsq{}}\PY{p}{)}

\PY{c+c1}{\PYZsh{} Create a figure}
\PY{n}{fig}\PY{p}{,} \PY{n}{presdistplot} \PY{o}{=} \PY{n}{plt}\PY{o}{.}\PY{n}{subplots}\PY{p}{(}\PY{p}{)}


\PY{c+c1}{\PYZsh{} Plot the line and add titles}
\PY{n}{plothusly}\PY{p}{(}\PY{n}{presdistplot}\PY{p}{,}
          \PY{n}{pressure\PYZus{}distribution}\PY{p}{[}\PY{l+s+s2}{\PYZdq{}}\PY{l+s+s2}{x}\PY{l+s+s2}{\PYZdq{}}\PY{p}{]}\PY{p}{,}
          \PY{o}{\PYZhy{}}\PY{n}{pressure\PYZus{}distribution}\PY{p}{[}\PY{l+s+s2}{\PYZdq{}}\PY{l+s+s2}{Cpl}\PY{l+s+s2}{\PYZdq{}}\PY{p}{]}\PY{p}{,}
          \PY{n}{xtitle} \PY{o}{=} \PY{l+s+s1}{\PYZsq{}}\PY{l+s+s1}{x/c}\PY{l+s+s1}{\PYZsq{}}\PY{p}{,}
          \PY{n}{ytitle}\PY{o}{=}\PY{l+s+s1}{\PYZsq{}}\PY{l+s+s1}{\PYZhy{}C\PYZdl{}\PYZus{}P\PYZdl{}}\PY{l+s+s1}{\PYZsq{}}\PY{p}{,}
          \PY{n}{title}\PY{o}{=}\PY{l+s+s2}{\PYZdq{}}\PY{l+s+s2}{Surface Pressure Distribution}\PY{l+s+s2}{\PYZdq{}}\PY{p}{,}
          \PY{n}{datalabel}\PY{o}{=}\PY{l+s+s1}{\PYZsq{}}\PY{l+s+s1}{Lower}\PY{l+s+s1}{\PYZsq{}}\PY{p}{,}
          \PY{n}{linestyle}\PY{o}{=}\PY{l+s+s1}{\PYZsq{}}\PY{l+s+s1}{\PYZhy{}}\PY{l+s+s1}{\PYZsq{}}\PY{p}{)} 

\PY{c+c1}{\PYZsh{} Plot the lower curve}
\PY{n}{plothus}\PY{p}{(}\PY{n}{presdistplot}\PY{p}{,}
        \PY{n}{pressure\PYZus{}distribution}\PY{p}{[}\PY{l+s+s1}{\PYZsq{}}\PY{l+s+s1}{x}\PY{l+s+s1}{\PYZsq{}}\PY{p}{]}\PY{p}{,}
        \PY{o}{\PYZhy{}}\PY{n}{pressure\PYZus{}distribution}\PY{p}{[}\PY{l+s+s1}{\PYZsq{}}\PY{l+s+s1}{Cpu}\PY{l+s+s1}{\PYZsq{}}\PY{p}{]}\PY{p}{,}
        \PY{n}{datalabel}\PY{o}{=}\PY{l+s+s1}{\PYZsq{}}\PY{l+s+s1}{Upper}\PY{l+s+s1}{\PYZsq{}}\PY{p}{,}
        \PY{n}{linestyle}\PY{o}{=}\PY{l+s+s1}{\PYZsq{}}\PY{l+s+s1}{\PYZhy{}}\PY{l+s+s1}{\PYZsq{}}\PY{p}{)}
\PY{n}{plt}\PY{o}{.}\PY{n}{legend}\PY{p}{(}\PY{n}{loc}\PY{o}{=}\PY{l+s+s1}{\PYZsq{}}\PY{l+s+s1}{best}\PY{l+s+s1}{\PYZsq{}}\PY{p}{)}
\end{Verbatim}
\end{tcolorbox}

            \begin{tcolorbox}[breakable, size=fbox, boxrule=.5pt, pad at break*=1mm, opacityfill=0]
\prompt{Out}{outcolor}{6}{\boxspacing}
\begin{Verbatim}[commandchars=\\\{\}]
<matplotlib.legend.Legend at 0x7fda7f278c10>
\end{Verbatim}
\end{tcolorbox}
        
    \begin{center}
    \adjustimage{max size={0.9\linewidth}{0.9\paperheight}}{output_11_1.png}
    \end{center}
    { \hspace*{\fill} \\}
    
    \hypertarget{surface-pressure-gradient-and-numeric-differentiation}{%
\subsubsection{3.2 \textbar{} Surface Pressure Gradient and Numeric
Differentiation}\label{surface-pressure-gradient-and-numeric-differentiation}}

\hypertarget{approach}{%
\paragraph{3.2.1 \textbar{} Approach}\label{approach}}

The goal in Problem 3.2 is to find the spatial gradient of the surface
pressure distribution, \(\frac{\partial C_P}{\partial \frac{x}{c}}\), by
using the data from the previous problem. The numerical derivative
itself can be calculated using a first-order forward derivative. The
``forward'' refers to the method, in that the local derivative is
calculated from index \(i\), and the next point in the index \(i+1\).
The data can then be plotted using matplotlib's pyplot library.

\hypertarget{equations}{%
\paragraph{3.2.2 \textbar{} Equations}\label{equations}}

Forward differentiation can be expressed as:

\[\begin{equation}
    (\frac{\partial C_P}{\partial \frac{x}{c}})_i = \frac{C_{P,i+1} - C_{P,i}}{\frac{x}{c}_{i+1} - \frac{x}{c}_i}
\end{equation}\]

\hypertarget{code-and-results}{%
\paragraph{3.2.3 \textbar{} Code and Results}\label{code-and-results}}

    \begin{tcolorbox}[breakable, size=fbox, boxrule=1pt, pad at break*=1mm,colback=cellbackground, colframe=cellborder]
\prompt{In}{incolor}{7}{\boxspacing}
\begin{Verbatim}[commandchars=\\\{\}]
\PY{c+c1}{\PYZsh{} Problem 3b}

\PY{c+c1}{\PYZsh{} Create placeholder vector for pressure gradient}
\PY{n}{pressure\PYZus{}gradient} \PY{o}{=} \PY{n}{np}\PY{o}{.}\PY{n}{zeros}\PY{p}{(}\PY{p}{(}\PY{n+nb}{len}\PY{p}{(}\PY{n}{pressure\PYZus{}distribution}\PY{p}{)}\PY{p}{,} \PY{l+m+mi}{1}\PY{p}{)}\PY{p}{)}

\PY{c+c1}{\PYZsh{} Determine gradient for each point for lower cp}
\PY{k}{for} \PY{n}{i} \PY{o+ow}{in} \PY{n+nb}{range}\PY{p}{(}\PY{n+nb}{len}\PY{p}{(}\PY{n}{pressure\PYZus{}distribution}\PY{p}{)}\PY{o}{\PYZhy{}}\PY{l+m+mi}{1}\PY{p}{)}\PY{p}{:}
    \PY{n}{pressure\PYZus{}gradient}\PY{p}{[}\PY{n}{i}\PY{p}{]} \PY{o}{=} \PY{p}{(}\PY{p}{(}\PY{n}{pressure\PYZus{}distribution}\PY{p}{[}\PY{l+s+s2}{\PYZdq{}}\PY{l+s+s2}{Cpl}\PY{l+s+s2}{\PYZdq{}}\PY{p}{]}\PY{p}{[}\PY{n}{i}\PY{o}{+}\PY{l+m+mi}{1}\PY{p}{]} 
                             \PY{o}{\PYZhy{}} \PY{n}{pressure\PYZus{}distribution}\PY{p}{[}\PY{l+s+s2}{\PYZdq{}}\PY{l+s+s2}{Cpl}\PY{l+s+s2}{\PYZdq{}}\PY{p}{]}\PY{p}{[}\PY{n}{i}\PY{p}{]}\PY{p}{)}
                            \PY{o}{/} \PY{p}{(}\PY{n}{pressure\PYZus{}distribution}\PY{p}{[}\PY{l+s+s1}{\PYZsq{}}\PY{l+s+s1}{x}\PY{l+s+s1}{\PYZsq{}}\PY{p}{]}\PY{p}{[}\PY{n}{i}\PY{o}{+}\PY{l+m+mi}{1}\PY{p}{]}
                               \PY{o}{\PYZhy{}} \PY{n}{pressure\PYZus{}distribution}\PY{p}{[}\PY{l+s+s1}{\PYZsq{}}\PY{l+s+s1}{x}\PY{l+s+s1}{\PYZsq{}}\PY{p}{]}\PY{p}{[}\PY{n}{i}\PY{p}{]}\PY{p}{)}\PY{p}{)}
    
\PY{c+c1}{\PYZsh{} Add to dataframe, reset buffer vector}
\PY{n}{pressure\PYZus{}distribution}\PY{p}{[}\PY{l+s+s1}{\PYZsq{}}\PY{l+s+s1}{Gradl}\PY{l+s+s1}{\PYZsq{}}\PY{p}{]} \PY{o}{=} \PY{n}{pressure\PYZus{}gradient}
\PY{n}{pressure\PYZus{}gradient} \PY{o}{*}\PY{o}{=} \PY{l+m+mi}{0}
 
\PY{c+c1}{\PYZsh{} Repeat last for loop for upper cp and add to dataframe}
\PY{k}{for} \PY{n}{i} \PY{o+ow}{in} \PY{n+nb}{range}\PY{p}{(}\PY{n+nb}{len}\PY{p}{(}\PY{n}{pressure\PYZus{}distribution}\PY{p}{)}\PY{o}{\PYZhy{}}\PY{l+m+mi}{1}\PY{p}{)}\PY{p}{:}
    \PY{n}{pressure\PYZus{}gradient}\PY{p}{[}\PY{n}{i}\PY{p}{]} \PY{o}{=} \PY{p}{(}\PY{p}{(}\PY{n}{pressure\PYZus{}distribution}\PY{p}{[}\PY{l+s+s2}{\PYZdq{}}\PY{l+s+s2}{Cpu}\PY{l+s+s2}{\PYZdq{}}\PY{p}{]}\PY{p}{[}\PY{n}{i}\PY{o}{+}\PY{l+m+mi}{1}\PY{p}{]} 
                              \PY{o}{\PYZhy{}} \PY{n}{pressure\PYZus{}distribution}\PY{p}{[}\PY{l+s+s2}{\PYZdq{}}\PY{l+s+s2}{Cpu}\PY{l+s+s2}{\PYZdq{}}\PY{p}{]}\PY{p}{[}\PY{n}{i}\PY{p}{]}\PY{p}{)}
                            \PY{o}{/} \PY{p}{(}\PY{n}{pressure\PYZus{}distribution}\PY{p}{[}\PY{l+s+s1}{\PYZsq{}}\PY{l+s+s1}{x}\PY{l+s+s1}{\PYZsq{}}\PY{p}{]}\PY{p}{[}\PY{n}{i}\PY{o}{+}\PY{l+m+mi}{1}\PY{p}{]}
                                \PY{o}{\PYZhy{}} \PY{n}{pressure\PYZus{}distribution}\PY{p}{[}\PY{l+s+s1}{\PYZsq{}}\PY{l+s+s1}{x}\PY{l+s+s1}{\PYZsq{}}\PY{p}{]}\PY{p}{[}\PY{n}{i}\PY{p}{]}\PY{p}{)}\PY{p}{)}

\PY{n}{pressure\PYZus{}distribution}\PY{p}{[}\PY{l+s+s1}{\PYZsq{}}\PY{l+s+s1}{Gradu}\PY{l+s+s1}{\PYZsq{}}\PY{p}{]} \PY{o}{=} \PY{n}{pressure\PYZus{}gradient}


\PY{c+c1}{\PYZsh{} Create plot}
\PY{n}{fig}\PY{p}{,} \PY{n}{presgradplot} \PY{o}{=} \PY{n}{plt}\PY{o}{.}\PY{n}{subplots}\PY{p}{(}\PY{p}{)}

\PY{n}{plothusly}\PY{p}{(}\PY{n}{presgradplot}\PY{p}{,} \PY{n}{pressure\PYZus{}distribution}\PY{p}{[}\PY{l+s+s2}{\PYZdq{}}\PY{l+s+s2}{x}\PY{l+s+s2}{\PYZdq{}}\PY{p}{]}\PY{p}{,}
          \PY{n}{pressure\PYZus{}distribution}\PY{p}{[}\PY{l+s+s2}{\PYZdq{}}\PY{l+s+s2}{Gradl}\PY{l+s+s2}{\PYZdq{}}\PY{p}{]}\PY{p}{,} \PY{n}{xtitle} \PY{o}{=} \PY{l+s+s1}{\PYZsq{}}\PY{l+s+s1}{x/c}\PY{l+s+s1}{\PYZsq{}}\PY{p}{,}
          \PY{n}{ytitle}\PY{o}{=}\PY{l+s+s1}{\PYZsq{}}\PY{l+s+s1}{dC\PYZdl{}\PYZus{}P\PYZdl{}/dx}\PY{l+s+s1}{\PYZsq{}}\PY{p}{,} \PY{n}{title}\PY{o}{=}\PY{l+s+s2}{\PYZdq{}}\PY{l+s+s2}{Surface Pressure Gradient}\PY{l+s+s2}{\PYZdq{}}\PY{p}{,}
          \PY{n}{datalabel}\PY{o}{=}\PY{l+s+s1}{\PYZsq{}}\PY{l+s+s1}{Lower}\PY{l+s+s1}{\PYZsq{}}\PY{p}{,} \PY{n}{linestyle}\PY{o}{=}\PY{l+s+s1}{\PYZsq{}}\PY{l+s+s1}{\PYZhy{}}\PY{l+s+s1}{\PYZsq{}}\PY{p}{)} 

\PY{n}{plothus}\PY{p}{(}\PY{n}{presgradplot}\PY{p}{,} \PY{n}{pressure\PYZus{}distribution}\PY{p}{[}\PY{l+s+s1}{\PYZsq{}}\PY{l+s+s1}{x}\PY{l+s+s1}{\PYZsq{}}\PY{p}{]}\PY{p}{,}
        \PY{n}{pressure\PYZus{}distribution}\PY{p}{[}\PY{l+s+s1}{\PYZsq{}}\PY{l+s+s1}{Gradu}\PY{l+s+s1}{\PYZsq{}}\PY{p}{]}\PY{p}{,} \PY{n}{datalabel}\PY{o}{=}\PY{l+s+s1}{\PYZsq{}}\PY{l+s+s1}{Upper}\PY{l+s+s1}{\PYZsq{}}\PY{p}{,}
        \PY{n}{linestyle}\PY{o}{=}\PY{l+s+s1}{\PYZsq{}}\PY{l+s+s1}{\PYZhy{}}\PY{l+s+s1}{\PYZsq{}}\PY{p}{)}
\PY{n}{plt}\PY{o}{.}\PY{n}{xlim}\PY{p}{(}\PY{p}{[}\PY{l+m+mi}{0}\PY{p}{,} \PY{l+m+mi}{1}\PY{p}{]}\PY{p}{)}
\PY{n}{plt}\PY{o}{.}\PY{n}{ylim}\PY{p}{(}\PY{p}{[}\PY{o}{\PYZhy{}}\PY{l+m+mi}{10}\PY{p}{,} \PY{l+m+mi}{10}\PY{p}{]}\PY{p}{)}
\PY{n}{plt}\PY{o}{.}\PY{n}{legend}\PY{p}{(}\PY{n}{loc}\PY{o}{=}\PY{l+s+s1}{\PYZsq{}}\PY{l+s+s1}{best}\PY{l+s+s1}{\PYZsq{}}\PY{p}{)}
\end{Verbatim}
\end{tcolorbox}

            \begin{tcolorbox}[breakable, size=fbox, boxrule=.5pt, pad at break*=1mm, opacityfill=0]
\prompt{Out}{outcolor}{7}{\boxspacing}
\begin{Verbatim}[commandchars=\\\{\}]
<matplotlib.legend.Legend at 0x7fda7f1f6c50>
\end{Verbatim}
\end{tcolorbox}
        
    \begin{center}
    \adjustimage{max size={0.9\linewidth}{0.9\paperheight}}{output_13_1.png}
    \end{center}
    { \hspace*{\fill} \\}
    
    \hypertarget{lift-curves-and-linear-interpolation}{%
\subsection{4 \textbar{} Lift Curves and Linear
Interpolation}\label{lift-curves-and-linear-interpolation}}

\hypertarget{lift-curve-and-excel-files}{%
\subsubsection{4.1 Lift Curve and Excel
Files}\label{lift-curve-and-excel-files}}

\hypertarget{approach}{%
\paragraph{4.1.1 \textbar{} Approach}\label{approach}}

The goal in Problem 4.1 is to plot a lift curve for a NACA 2412 airfoil,
using data provided in a .xlsx file. The file is read through pandas'
read\_excel function, and then the data is plotted using matplotlib's
pyplot library, in a manner similar to the previous problems.

\hypertarget{equations}{%
\paragraph{4.1.2 \textbar{} Equations}\label{equations}}

No equations were used in this problem, as it is purely a plotting
excersize.

\hypertarget{code-and-results}{%
\paragraph{4.1.3 \textbar{} Code and Results}\label{code-and-results}}

    \begin{tcolorbox}[breakable, size=fbox, boxrule=1pt, pad at break*=1mm,colback=cellbackground, colframe=cellborder]
\prompt{In}{incolor}{8}{\boxspacing}
\begin{Verbatim}[commandchars=\\\{\}]
\PY{c+c1}{\PYZsh{} Problem 4a}

\PY{n}{naca\PYZus{}2412\PYZus{}lift\PYZus{}curve} \PY{o}{=} \PY{n}{pd}\PY{o}{.}\PY{n}{read\PYZus{}excel}\PY{p}{(}\PY{l+s+s2}{\PYZdq{}}\PY{l+s+s2}{Data/naca2412\PYZus{}LiftCurve.xlsx}\PY{l+s+s2}{\PYZdq{}}\PY{p}{)}
\PY{n}{fig}\PY{p}{,} \PY{n}{liftplot} \PY{o}{=} \PY{n}{plt}\PY{o}{.}\PY{n}{subplots}\PY{p}{(}\PY{p}{)}
\PY{n}{plothusly}\PY{p}{(}\PY{n}{liftplot}\PY{p}{,} \PY{n}{naca\PYZus{}2412\PYZus{}lift\PYZus{}curve}\PY{p}{[}\PY{l+s+s1}{\PYZsq{}}\PY{l+s+s1}{alpha}\PY{l+s+s1}{\PYZsq{}}\PY{p}{]}\PY{p}{,} \PY{n}{naca\PYZus{}2412\PYZus{}lift\PYZus{}curve}\PY{p}{[}\PY{l+s+s2}{\PYZdq{}}\PY{l+s+s2}{Cl}\PY{l+s+s2}{\PYZdq{}}\PY{p}{]}\PY{p}{,}
          \PY{n}{xtitle}\PY{o}{=}\PY{l+s+sa}{r}\PY{l+s+s1}{\PYZsq{}}\PY{l+s+s1}{Angle of Attack \PYZdl{}}\PY{l+s+s1}{\PYZbs{}}\PY{l+s+s1}{alpha\PYZdl{}}\PY{l+s+s1}{\PYZsq{}}\PY{p}{,} \PY{n}{ytitle}\PY{o}{=}\PY{l+s+sa}{r}\PY{l+s+s1}{\PYZsq{}}\PY{l+s+s1}{C\PYZdl{}\PYZus{}l\PYZdl{}}\PY{l+s+s1}{\PYZsq{}}\PY{p}{,}
          \PY{n}{title}\PY{o}{=}\PY{l+s+s1}{\PYZsq{}}\PY{l+s+s1}{NACA 2412 Lift Curve}\PY{l+s+s1}{\PYZsq{}}\PY{p}{,} \PY{n}{marker}\PY{o}{=}\PY{l+s+s1}{\PYZsq{}}\PY{l+s+s1}{o}\PY{l+s+s1}{\PYZsq{}}\PY{p}{)}
\end{Verbatim}
\end{tcolorbox}

            \begin{tcolorbox}[breakable, size=fbox, boxrule=.5pt, pad at break*=1mm, opacityfill=0]
\prompt{Out}{outcolor}{8}{\boxspacing}
\begin{Verbatim}[commandchars=\\\{\}]
[<matplotlib.lines.Line2D at 0x7fda7f264890>]
\end{Verbatim}
\end{tcolorbox}
        
    \begin{center}
    \adjustimage{max size={0.9\linewidth}{0.9\paperheight}}{output_15_1.png}
    \end{center}
    { \hspace*{\fill} \\}
    
    \hypertarget{linear-interpolation}{%
\subsubsection{4.2 \textbar{} Linear
Interpolation}\label{linear-interpolation}}

\hypertarget{approach}{%
\paragraph{4.2.1 \textbar{} Approach}\label{approach}}

The goal in Problem 4.2 is to use linear interpolation to find the
C\(_l\) value at \(\alpha = 5.65\) degrees. This can be done with
numpy's interp function, which linearly interpolates using two sets of
input data. Using the given alpha value, and indexing using the
\(\alpha\) column, the interpolated value for C\(_l\) can be found, and
then printed to console using an f-string.

\hypertarget{equations}{%
\paragraph{4.2.2 \textbar{} Equations}\label{equations}}

There were no equations used in this problem, as it is primarily a
tutorial on numpy's interpolation function.

\hypertarget{code-and-results}{%
\paragraph{4.2.3 \textbar{} Code and Results}\label{code-and-results}}

The interpolated value is calculated, and then printed to console.

    \begin{tcolorbox}[breakable, size=fbox, boxrule=1pt, pad at break*=1mm,colback=cellbackground, colframe=cellborder]
\prompt{In}{incolor}{9}{\boxspacing}
\begin{Verbatim}[commandchars=\\\{\}]
\PY{n}{c\PYZus{}l\PYZus{}interpolated} \PY{o}{=} \PY{n}{np}\PY{o}{.}\PY{n}{interp}\PY{p}{(}\PY{l+m+mf}{5.65}\PY{p}{,} \PY{n}{naca\PYZus{}2412\PYZus{}lift\PYZus{}curve}\PY{p}{[}\PY{l+s+s2}{\PYZdq{}}\PY{l+s+s2}{alpha}\PY{l+s+s2}{\PYZdq{}}\PY{p}{]}\PY{p}{,}
                             \PY{n}{naca\PYZus{}2412\PYZus{}lift\PYZus{}curve}\PY{p}{[}\PY{l+s+s2}{\PYZdq{}}\PY{l+s+s2}{Cl}\PY{l+s+s2}{\PYZdq{}}\PY{p}{]}\PY{p}{)}
\PY{n}{string} \PY{o}{=} \PY{l+s+sa}{rf}\PY{l+s+s2}{\PYZdq{}}\PY{l+s+s2}{Interpolated C\PYZus{}l at alpha = 5.65 deg is }\PY{l+s+si}{\PYZob{}}\PY{n}{c\PYZus{}l\PYZus{}interpolated}\PY{l+s+si}{\PYZcb{}}\PY{l+s+s2}{\PYZdq{}}
\PY{n+nb}{print}\PY{p}{(}\PY{n}{string}\PY{p}{)}
\end{Verbatim}
\end{tcolorbox}

    \begin{Verbatim}[commandchars=\\\{\}]
Interpolated C\_l at alpha = 5.65 deg is 0.8695250000000001
    \end{Verbatim}

    \hypertarget{linear-algebra}{%
\subsection{5 \textbar Linear Algebra}\label{linear-algebra}}

\hypertarget{approach}{%
\paragraph{5.1 \textbar{} Approach}\label{approach}}

The goal for Problem 5.1 is to use numpy's linalg.solve() function to
solve for a system of equations. The system is placed in matrix form,
and placed into a numpy array. The solution is similarly placed into
matrix form, and placed in a numpy array. The system is then solved
using numpy's linalg.solve() function, and the solution vector is
printed to the terminal.

\hypertarget{equations}{%
\paragraph{5.2 \textbar{} Equations}\label{equations}}

The system of equation can be expressed in the form:

\[\begin{equation}
    A \lambda = b
\end{equation}\]

as:

\[\begin{equation}
    \begin{bmatrix}
        1&2&3&4\\
        3&2&-2&3\\
        0&1&1&0\\
        2&1&1&-2\\
    \end{bmatrix} \begin{bmatrix}
    w\\
    x\\
    y\\
    z\\
    \end{bmatrix} = \begin{bmatrix}
        12\\
        10\\
        -1\\
        -5\\
    \end{bmatrix}
\end{equation}\]

and solved as:

\[\begin{equation}
    \lambda = A^{-1}b
\end{equation}\]

\hypertarget{code-and-results}{%
\paragraph{5.3 \textbar{} Code and Results}\label{code-and-results}}

    \begin{tcolorbox}[breakable, size=fbox, boxrule=1pt, pad at break*=1mm,colback=cellbackground, colframe=cellborder]
\prompt{In}{incolor}{10}{\boxspacing}
\begin{Verbatim}[commandchars=\\\{\}]
\PY{n}{A} \PY{o}{=} \PY{n}{np}\PY{o}{.}\PY{n}{array}\PY{p}{(}\PY{p}{[}\PY{p}{[}\PY{l+m+mi}{1}\PY{p}{,} \PY{l+m+mi}{2}\PY{p}{,} \PY{l+m+mi}{3}\PY{p}{,} \PY{l+m+mi}{4}\PY{p}{]}\PY{p}{,}
              \PY{p}{[}\PY{l+m+mi}{3}\PY{p}{,} \PY{l+m+mi}{2}\PY{p}{,} \PY{o}{\PYZhy{}}\PY{l+m+mi}{2}\PY{p}{,} \PY{l+m+mi}{3}\PY{p}{]}\PY{p}{,}
              \PY{p}{[}\PY{l+m+mi}{0}\PY{p}{,} \PY{l+m+mi}{1}\PY{p}{,} \PY{l+m+mi}{1}\PY{p}{,} \PY{l+m+mi}{0}\PY{p}{]}\PY{p}{,}
              \PY{p}{[}\PY{l+m+mi}{2}\PY{p}{,} \PY{l+m+mi}{1}\PY{p}{,} \PY{l+m+mi}{1}\PY{p}{,} \PY{o}{\PYZhy{}}\PY{l+m+mi}{2}\PY{p}{]}\PY{p}{]}\PY{p}{)}
\PY{n}{b} \PY{o}{=} \PY{n}{np}\PY{o}{.}\PY{n}{array}\PY{p}{(}\PY{p}{[}\PY{p}{[}\PY{l+m+mi}{12}\PY{p}{]}\PY{p}{,}
              \PY{p}{[}\PY{l+m+mi}{10}\PY{p}{]}\PY{p}{,}
              \PY{p}{[}\PY{o}{\PYZhy{}}\PY{l+m+mi}{1}\PY{p}{]}\PY{p}{,}
              \PY{p}{[}\PY{o}{\PYZhy{}}\PY{l+m+mi}{5}\PY{p}{]}\PY{p}{]}\PY{p}{)}
\PY{n}{x} \PY{o}{=} \PY{n}{np}\PY{o}{.}\PY{n}{linalg}\PY{o}{.}\PY{n}{solve}\PY{p}{(}\PY{n}{A}\PY{p}{,} \PY{n}{b}\PY{p}{)}
\PY{n+nb}{print}\PY{p}{(}\PY{l+s+sa}{f}\PY{l+s+s1}{\PYZsq{}}\PY{l+s+s1}{x = }\PY{l+s+si}{\PYZob{}}\PY{n}{x}\PY{l+s+si}{\PYZcb{}}\PY{l+s+s1}{\PYZsq{}}\PY{p}{)}
\end{Verbatim}
\end{tcolorbox}

    \begin{Verbatim}[commandchars=\\\{\}]
x = [[ 1.15384615]
 [-1.23076923]
 [ 0.23076923]
 [ 3.15384615]]
    \end{Verbatim}


    % Add a bibliography block to the postdoc
    
    
    
\end{document}
